%% appendices for thesis/proposal

\section*{Appendix A:  Algorithms Pseudo Code}
Define $\small{\mathcal{T}^0}$ as the initialized tree, $n$ as the number of iterations of each MCMC chain, $\underline{p}^0$ as the initialized probability weights and $\underline{\alpha}^\prime$ as the initialized pseudo-counts for the splits in each dimension, $s$ are the observed split counts from each sampled tree. As defaults we take $\underline{p}^0\propto \underline{1}$, and  $\underline{\alpha}^\prime \propto \underline{1}$. Also $N$ is the tree likelihood of the new or proposed tree and $O$ is the old tree likelihood, both are on the log scale. Finally, $b$ denotes the number of terminal nodes in the current ($\mathcal{T}^i$) decision tree. 

\begin{algorithm}[H]
			%\dontprintsemicolon
			\KwData{ $n, \underline{p}^0,\mathcal{T}^0, \alpha^0, C$}
\For{ $i=2, \dots, \# \text{samples}$}{
	$ X \sim Discrete\_Uniform\left(1,5\right);$\\
\CASE {(X=1)} $ \mathcal{T}^\prime \GETS Grow(\mathcal{T}^i);$\\
\CASE {(X=2)} $ \mathcal{T}^\prime \GETS Prune(\mathcal{T}^i);$\\
\CASE {(X=3)} $ \mathcal{T}^\prime \GETS Change(\mathcal{T}^i);$\\
\CASE {(X=4)} $ \mathcal{T}^\prime \GETS Swap(\mathcal{T}^i);$\\
\CASE {(X=5)} $ \mathcal{T}^\prime \GETS Rotate(\mathcal{T}^i);$\\
\IF \{$log(U) < log(R)$\}  $\ \ \mathcal{T}^{i} \GETS \mathcal{T}^\prime; $\\
$\textbf{else}  \mathcal{T}^{i} \GETS \mathcal{T}^{i-1};$\\
$ \underline{\alpha}^{i} \GETS \underline{\alpha}^{i-1} + \widetilde{\alpha}\underline{s};$\\
$ \underline{p}^i \GETS Dirichlet(\underline{\alpha}^{i});$ \\
 $\widetilde{\alpha} \GETS C \sum_{j=1}^d\alpha_j / \sum_{i,j}s_{ij};$ 
}
\caption{DiVaS sampler.}\label{psd:DiVaS_sampler}
\end{algorithm}

The second pseudo-code listing contains the simple sampler approach from Chapter \ref{sec:ALN_chapter}, the notation is similar to the first pseudocode.  

\begin{algorithm}[H]
			%\dontprintsemicolon
			\KwData{ $n, \underline{p}^0,\mathcal{T}^0, \alpha^0$}
\For{ $i=2, \dots, \# \text{samples}$}{
	$\ \ X \GETS Discrete\_Uniform\left(1,5\right);$\\
$\CASE {(X=1)} \ \ \mathcal{T}^\prime \GETS Grow(\mathcal{T}^i);$\\
$\CASE {(X=2)} \ \ \mathcal{T}^\prime \GETS Prune(\mathcal{T}^i);$\\
$\CASE {(X=3)} \ \ \mathcal{T}^\prime \GETS Change(\mathcal{T}^i);$\\
$\CASE {(X=4)} \ \ \mathcal{T}^\prime \GETS Swap(\mathcal{T}^i);$\\
$\CASE {(X=5)} \ \ \mathcal{T}^\prime \GETS Rotate(\mathcal{T}^i);$\\
$\ \ N \GETS  log(\Pr(D \vert \mathcal{T}^\prime)); $\\
$\ \ O \GETS log(\Pr(D \vert \mathcal{T}^{i-1}));$\\
$\CASE {(X=1)} \ \ log(R) \GETS N-O+log(b);$\\
$\CASE {(X=2)} \ \ log(R) \GETS N-O+log(b+1);$\\ 
$\CASE {(X=3)} \ \ log(R) \GETS N-O+log(p_j)\ -\ log(p_{j^\prime});$\\
$\CASE {(X=4 \OR X=5)} \ \ log(R) \GETS  N\ -\ O;$\\
$\ \ U\GETS Continuous\_Uniform(0,1);$\\
$\IF\{ log(U) < log(R)\}  \ \ \mathcal{T}^{i} \GETS \mathcal{T}^\prime; $\\
$\textbf{else} \ \ \mathcal{T}^{i} \GETS \mathcal{T}^{i-1};$\\
\For{ $j = 1,\dots, d$}{
$u_j \GETS Unif(0, \exp{(-c_js_j-\mu_j)^2/2\sigma_j^2}) $\\
$c_j \GETS Unif(max(-a,-\sqrt{-2log(u_j)}), min(a, \sqrt{-2log(u_j)}))$\\
$\sigma_j^2 \GETS Inv-Gamma(2a_j +2, ( (c_js_j-\mu_j)^2+(\mu_j-\mu_j^p)^2 )/2+ b_j)$\\
$\mu_j \GETS N[c_js_j+\mu_j^p, \sigma^2_j]$\\
$p_j\GETS\exp(\mu_j)/(1+\sum_{k=1}^d\exp(\mu_k))$\\
}
$p_{d+1}\GETS 1-\sum_{k=1}^dp_k$
}
\caption{Simple sampler.}\label{psd:simple_sampler}
\end{algorithm}



%\begin{framed}
%\vspace{-.2in}
%\begin{pseudocode}{Simple Sampler}{}
%\FOR i \GETS 1 \TO n \\
%\ \ X \GETS Discrete\_Uniform\left(1,5\right);\\
%\CASE {(X=1)} \ \ \mathcal{T}^\prime \GETS Grow(\mathcal{T}^i);\\
%\CASE {(X=2)} \ \ \mathcal{T}^\prime \GETS Prune(\mathcal{T}^i);\\
%\CASE {(X=3)} \ \ \mathcal{T}^\prime \GETS Change(\mathcal{T}^i);\\
%\CASE {(X=4)} \ \ \mathcal{T}^\prime \GETS Swap(\mathcal{T}^i);\\
%\CASE {(X=5)} \ \ \mathcal{T}^\prime \GETS Rotate(\mathcal{T}^i);\\
%\ \ N \GETS  log(\Pr(D \vert \mathcal{T}^\prime)); \\
%\ \ O \GETS log(\Pr(D \vert \mathcal{T}^{i-1}));\\
%\CASE {(X=1)} \ \ log(R) \GETS N-O+log(b);\\
%\CASE {(X=2)} \ \ log(R) \GETS N-O+log(b+1);\\ 
%\CASE {(X=3)} \ \ log(R) \GETS N-O+log(p_j)\ -\ log(p_{j^\prime});\\
%\CASE {(X=4 \OR X=5)} \ \ log(R) \GETS  N\ -\ O;\\
%\ \ U\GETS Continuous\_Uniform(0,1);\\
%\IF \{log(U) < log(R)\}  \ \ \mathcal{T}^{i} \GETS \mathcal{T}^\prime; \\
%\textbf{else} \ \ \mathcal{T}^{i} \GETS \mathcal{T}^{i-1};\\
%\FOR j \GETS 1\TO d\\
%u_j \GETS Unif(0, \exp{(-c_js_j-\mu_j)^2/2\sigma_j^2}) \\
%c_j \GETS Unif(max(-a,-\sqrt{-2log(u_j)}), min(a, \sqrt{-2log(u_j)}))\\
%\sigma_j^2 \GETS Inv-Gamma(2a_j +2, ( (c_js_j-\mu_j)^2+(\mu_j-\mu_j^p)^2 )/2+ b_j)\\
%\mu_j \GETS N[c_js_j+\mu_j^p, \sigma^2_j]\\
%p_j\GETS\exp(\mu_j)/(1+\sum_{k=1}^d\exp(\mu_k))\\
%\textbf{EndFor}\\
%p_{d+1}\GETS 1-\sum_{k=1}^dp_k\\
%\end{pseudocode}
%\vspace{-.3in}
%\end{framed}

\section*{Appendix B: Non-negative Garrote Solutions when $X^TX = I$.}

In this section we derive the non-negative garrote estimators under orthogonal designs. Recall a design matrix $X$ is called orthogonal, or more properly, orthonormal, if $X^TX=I$. This implies that $\sum_ix_{ij}^2=1$ and $\sum_ix_{ij}x_{ik}=0$ for $j\neq k$. 

Recall the non-negative garrote objective function is 

\begin{equation}\label{eqn:nn_garrote_obj}
\underset{\forall j: c_j\geq 0}{argmin}\ \ \sum_{i=1}^n(y_i -\sum_{j=1}^dc_j\hat{\beta_j}x_{ij})^2 + \lambda\sum_{j=1}^dc_j,
\end{equation}    

Now for simplicity of exposition we will take $d=2$ and also assume the $y_i$ have had their mean subtracted from each observation, removing the intercept term from the model. This gives us the objective 

\begin{equation}\label{eqn:nn_garroted=2}
\underset{\forall j: c_j\geq 0}{argmin}\ \ \underbrace{\frac{1}{2}\sum_{i=1}^n(y_i -c_1\hat{\beta}_1x_{i1}-c_2\hat{\beta}_2x_{i2})^2 + \lambda(c_1+c_2)}_{=f(c_1,c_2)},
\end{equation}    

Now we optimize over $c_j$ by taking derivatives. This results in a system of 2 linear equations, the details follow: 

\begin{align}
\frac{\partial f}{\partial c_1} &= \sum_i(y_i-c_1\hat{\beta}_1x_{i1}-c_2\hat{\beta}_2x_{i2})(-\hat{\beta}_1x_{i1}) + \lambda \overset{\text{set}}{=} 0\\
\frac{\partial f}{\partial c_2} &= \sum_i(y_i-c_1\hat{\beta}_1x_{i1}-c_2\hat{\beta}_2x_{i2})(-\hat{\beta}_2x_{i2}) +\lambda \overset{\text{set}}{=} 0
\end{align}

Multiplying the extra terms through gives us the equations

\begin{align}
\frac{\partial f}{\partial c_1} &=  \sum_i(-y_i\hat{\beta}_1x_{i1}+c_1\hat{\beta}_1x_{i1}\hat{\beta}_1x_{i1}+c_2\hat{\beta}_2x_{i2}\hat{\beta}_1x_{i1})+\lambda \overset{\text{set}}{=} 0\\
\frac{\partial f}{\partial c_2} &= \sum_i(-y_i\hat{\beta}_2x_{i2}+c_1\hat{\beta}_1x_{i1}\hat{\beta}_2x_{i2}+c_2\hat{\beta}_2\hat{\beta}_2x_{i2}x_{i2}) +\lambda\overset{\text{set}}{=} 0
\end{align}

Now after some algebra we get the analogs to the ``normal'' equations in least squares

\begin{align}
  \hat{\beta}_1\sum_iy_ix_{i1}&=\sum_i(c_1\hat{\beta}_1x_{i1}^2+c_2\hat{\beta}_2x_{i1}x_{i2}) +\lambda\\
 \hat{\beta}_2\sum_iy_ix_{i2}&=\sum_i(c_1\hat{\beta}_1x_{i1}x_{i2}+c_2\hat{\beta}_2x^2_{i2})+\lambda 
\end{align}

Now applying the sum through to the RHS of both equations and applying the orthonormal conditions we get

\begin{align}
  \hat{\beta}_1\sum_iy_ix_{i1}&=c_1\hat{\beta}_1^2+\lambda \\
 \hat{\beta}_2\sum_iy_ix_{i2}&=c_2\hat{\beta}_2^2 +\lambda
\end{align}

The orthonormal conditions imply $\sum_ix_{ij}^2=1$, so we divide the $\sum_i y_i x_{ij}$ terms by this ``1'' term. Noting that if the $x_{ij}$'s are centered about their means we have

\begin{equation}
\hat{\beta}_j = \frac{\sum_i x_{ij} y_i}{\sum_ix_{ij}^2} 
\end{equation}

Solving for $c_1$ and $c_2$ gives the equations 

\begin{align}
 1- \frac{\lambda}{\hat{\beta}_1^2}&=c_1 \\
  1-\frac{\lambda}{\hat{\beta}_2^2}&=c_2 
\end{align}

and noting that $c_j \geq 0$ implies we take the positive part.  These are the closed form solutions given in \cite{breiman1995better}. The reader can now easily generalize to the case with $d>2$ covariates. $\square$

The interested reader may work out the analogous results for the constraint $\sum_{j=1}^dc_j^2 \leq s$ for some constant $s$. The closed form solution in the orthonormal $X$ case is also given in \cite{breiman1995better}.  

\section*{Appendix C}

Using the result from Villa and Escobar \cite{villa2006using},  which states. 

\noindent \textbf{Theorem}\\

Suppose $M_{x\vert y}(t)=C_1(t)\exp[C_2(t)Y]$ and there exists a $\delta>0$ such that for $t\in(-\delta, \delta)$, $\vert C_i(t) \vert < \infty$  and $\vert M_y(C_2(t))\vert < \infty$ assuming $M_y(t)$ exists, then $M_x(t) =C_1(t)M_y(C_2(t))$. In more common moment generating function (MGF) \newabbrev{abbrev:MGF} notation we have $M_x(t) = \mathbb{E}_y(M_{x\vert y}(t))$. 
See Villa and Escobar for the proof. 

We an use this result to understand the mixture of normals. To do this we recall the forms of the MGFs for three common distributions: 

\begin{equation}\label{eqn:normal_mgf}
M(t) = \exp\left(\mu t + \frac{\sigma^2t^2}{2}\right)
\end{equation}

\begin{equation}\label{eqn:exponential_mgf}
M(t) =\frac{1}{1-\lambda t}
\end{equation}

\begin{equation}\label{eqn:laplace_mgf}
M(t) = \frac{ \exp\left(\mu t\right) }{1-b^2t^2}
\end{equation}

In MGF's \ref{eqn:normal_mgf}-\ref{eqn:laplace_mgf}, the means of the densities are $\mu$, $\lambda$, and $\mu$ respectively. Also, the variances of the densities are $\sigma^2$, $\lambda^2$, and $b^2$.  

Now applying the Theorem of Villa and Escobar we have 

\begin{align*}
\mathbb{E}\left(M_{x\vert v}(t)\right)&=\mathbb{E}\left(\exp\left(\mu t +2v\sigma^2t^2  \right)\right) \\
&= \int_0^\infty \exp\left(-v +\mu t +2v\sigma^2t^2  \right)dv\\
&= \exp(\mu t) \int_0^\infty \exp\left(-v +2v\sigma^2t^2  \right)dv\\
&= \exp(\mu t) \int_0^\infty \exp\left(-v(1-2\sigma^2t^2)\right)dv\\
&= \frac{\exp(\mu t) }{1-2\sigma^2t^2} \underbrace{\int_0^\infty(1-2\sigma^2t^2)\exp\left(-v(1-2\sigma^2t^2)\right)dv}_{=1, \text{  because it is an exponential pdf}}\\
&= \frac{\exp(\mu t) }{1-2\sigma^2t^2}
\end{align*}

The result that the Laplace distribution is a normal scale mixture of exponentials is shown. 
